\pdfoutput=1
% v. 0.01 - 23/04/2012 - initial version

\documentclass[a4paper,11pt]{article}

\usepackage{fullpage}
\usepackage{graphicx}
\usepackage{hyperref}
\usepackage{listings}
\usepackage{dsfont} % some math symbols
\usepackage{courier} % ttfamily
\usepackage[utf8]{inputenc}

\renewcommand{\familydefault}{\sfdefault}
\newcommand{\eg}{\textsl{e.g}}

\lstset{language=C}
\lstset{captionpos=b,basicstyle=\ttfamily,frame=lb,mathescape=true}

\title{Introduction to \texttt{libQRNG} library}
\author{Jaros{\l}aw A. Miszczak\\
Institute of Theoretical and Applied Informatics, Polish Academy of Sciences\\
Baltycka 5, 44-100 Gliwice, Poland}
\date{23/04/2012 (v. 0.01)}

\begin{document}

\maketitle

%%%%%%%%%%%%%%%%%%%%%%%%%%%%%%%%%%%%%%%%%%%%%%%%%%%%%%%%%%%%%%%%%%%%%%%%%%%%%%%%
\section{Establising connection with the QRGN server}
%%%%%%%%%%%%%%%%%%%%%%%%%%%%%%%%%%%%%%%%%%%%%%%%%%%%%%%%%%%%%%%%%%%%%%%%%%%%%%%%

In order to retrive data from the QRNG service provided by the PicoQuant and the
Department of Physics of Humboldt University, it is necessary to register on the
service web page~\cite{qrng-www}. In order to establish a connection with the server one can use
one of the function
\begin{itemize}
    \item \lstinline{qrng_connect(qrng_username, qrng_password)} -- establish a
      connection with the QRNG service (qrng.physik.hu-berlin.de:4499),

    \item \lstinline{qrng_connect_SSL(qrng_username, qrng_password)} --
      establish a secure connection with the QRNG service using SSL.
\end{itemize}
Both functions require username and password provided during the registration.
The connection is closed using \lstinline{qrng_disconnect()} function.
\begin{itemize}
	\item \lstinline{qrng_disconnect()} -- close the connection with the QRNG
	server.
\end{itemize}

%%%%%%%%%%%%%%%%%%%%%%%%%%%%%%%%%%%%%%%%%%%%%%%%%%%%%%%%%%%%%%%%%%%%%%%%%%%%%%%%
\section{Retriving random data}
%%%%%%%%%%%%%%%%%%%%%%%%%%%%%%%%%%%%%%%%%%%%%%%%%%%%%%%%%%%%%%%%%%%%%%%%%%%%%%%%
QRNG service allows to retrive random data in the form of integer number, double precision numbers or byte


\begin{thebibliography}{2}
    \bibitem{qrng-www} QRNG Service, \url{http://qrng.physik.hu-berlin.de}

    \bibitem{wahl11ultrafast} M.~Wahl, M. Leifgen, M. Berlin, T. R\"ohlicke,
       H.-J. Rahn, and O. Benson, \emph{An ultrafast quantum random number
       generator with provably bounded output bias based on photon arrival time
       measurements}, Appl. Phys. Lett. 98, 171105 (2011)
       \url{http://dx.doi.org/10.1063/1.3578456}
\end{thebibliography}

\end{document}
