\pdfoutput=1
% v. 0.01 - 15/02/2011 - Jarek - installation instructions written during tests
% v. 0.02 - 23/03/2011 - Jarek - minor improvements
% v. 0.03 - 23/04/2012 - Jarek - style and layout improvements
% v. 0.04 - 01/05/2012 - Iza - minor spelling and grammar corrections
% v. 0.05 - 27/06/2012 - Jarek - remarks concerning RedHad/Fedora-based systems

\documentclass[a4paper,11pt]{article}

\usepackage{fullpage}
\usepackage{graphicx}
\usepackage{hyperref}
\usepackage{listings}
\usepackage{dsfont} % some math symbols
\usepackage{courier} % ttfamily
\usepackage[utf8]{inputenc}

\renewcommand{\familydefault}{\sfdefault}
\newcommand{\eg}{\textsl{e.g}}

\title{Installation of Quantis drivers and libraries}
\author{Jaros{\l}aw A. Miszczak\\
Institute of Theoretical and Applied Informatics, Polish Academy of Sciences\\
Baltycka 5, 44-100 Gliwice, Poland}
\date{27/06/2012 (v.0.05)}

\providecommand{\note}[1]{\noindent\textbf{\underline{Note}}: #1\vspace{6pt}}
\newcommand{\QuantisDistVersion}{Quantis-v11.12.13}

\begin{document}

\maketitle

\begin{abstract}
The following instructions provide an overview of the installation procedure for
device drivers and libraries required to use Quantis quantum random number
generator on Linux-based systems. Please consult the documentation provided with
the Quantis software package for more information.
\end{abstract}

%%%%%%%%%%%%%%%%%%%%%%%%%%%%%%%%%%%%%%%%%%%%%%%%%%%%%%%%%%%%%%%%%%%%%%%%%%%%%%%%
\section{Device drivers}
%%%%%%%%%%%%%%%%%%%%%%%%%%%%%%%%%%%%%%%%%%%%%%%%%%%%%%%%%%%%%%%%%%%%%%%%%%%%%%%%

In the following instructions \texttt{\QuantisDistVersion } should be replaced
by the appropriate directory depending on the version of Quantis software
package you are using during the installation.

One should also note that most of the command issued during the installations
process require super-user (root) privileges. On most modern Linux systems
this can be achieved using \texttt{sudo} or \texttt{su} commands.

%%%%%%%%%%%%%%%%%%%%%%%%%%%%%%%%%%%%%%%%%%%%%%%%%%%%%%%%%%%%%%%%%%%%%%%%%%%%%%%%
\subsection{PCI and PCI-express version}
%%%%%%%%%%%%%%%%%%%%%%%%%%%%%%%%%%%%%%%%%%%%%%%%%%%%%%%%%%%%%%%%%%%%%%%%%%%%%%%%
\begin{enumerate}
    \item Install module-assistant package\\
    \texttt{apt-get install module-assistant}
    \note{On RedHat/Fedora-based systems you need to install
    \texttt{kernel-devel} package and collection of development tools}\\ 
    \texttt{yum install kernel-devel} \\
    \texttt{yum groupinstall "Development Tools"}

    \item Prepare your system for module compilation\\ \texttt{m-a prepare}

    \item Unpack the Quantis software package and go to\\
    \texttt{\QuantisDistVersion /Drivers/Unix/QuantisPci}.

    \item Compile the module\\ \texttt{make}

    \item Install the module\\ \texttt{make install}\\ and check if the module
    has been installed properly\\ \texttt{find /lib/modules/ -name
    quantis\_pci.ko}

    \item Make sure that the driver will be loaded during the next boot \\
    \texttt{echo "quantis\_pci" >> /etc/modules}

    \item Follow the instructions in Section~\ref{sec:permissions}.

\end{enumerate}

%%%%%%%%%%%%%%%%%%%%%%%%%%%%%%%%%%%%%%%%%%%%%%%%%%%%%%%%%%%%%%%%%%%%%%%%%%%%%%%%
\subsection{USB version}
%%%%%%%%%%%%%%%%%%%%%%%%%%%%%%%%%%%%%%%%%%%%%%%%%%%%%%%%%%%%%%%%%%%%%%%%%%%%%%%%
\begin{enumerate}
    \item Install \texttt{libusb} and \texttt{libusb-dev} packages\\ 
    \texttt{apt-get install libusb-1.0-0 libusb-1.0-0-dev}

    \item Check your device with \texttt{lsusb}\\
    \texttt{lsusb -d 0aba:0102 -v} or \texttt{lsusb -v | grep Quantis}

    \item Follow the instructions in Section~\ref{sec:permissions}.
\end{enumerate}


%%%%%%%%%%%%%%%%%%%%%%%%%%%%%%%%%%%%%%%%%%%%%%%%%%%%%%%%%%%%%%%%%%%%%%%%%%%%%%%%
\subsection{Device permissions}\label{sec:permissions}
%%%%%%%%%%%%%%%%%%%%%%%%%%%%%%%%%%%%%%%%%%%%%%%%%%%%%%%%%%%%%%%%%%%%%%%%%%%%%%%%
\begin{enumerate}
    \item Make sure that the \texttt{plugdev} group exists\\
    \texttt{cat /etc/group | grep plugdev}
    \item Add the user that will be permitted to use Quantis to the
    \texttt{plugdev} group\\ \texttt{usermod -G plugdev -a LOGIN},\\ where
    \texttt{LOGIN} is the login name of the user
    \item Copy file \texttt{\QuantisDistVersion
    /Drivers/Unix/idq-quantis.rules},\\ to \texttt{/etc/udev/rules.d/} directory
    and reload the UDEV rules\\ \texttt{udevadm control --reload-rules}
\end{enumerate}

%%%%%%%%%%%%%%%%%%%%%%%%%%%%%%%%%%%%%%%%%%%%%%%%%%%%%%%%%%%%%%%%%%%%%%%%%%%%%%%%
\section{Libraries}
%%%%%%%%%%%%%%%%%%%%%%%%%%%%%%%%%%%%%%%%%%%%%%%%%%%%%%%%%%%%%%%%%%%%%%%%%%%%%%%%

\begin{enumerate}
    \item Unpack distribution files from
    \texttt{\QuantisDistVersion /Packages/Linux} to a selected directory, \eg\
    \texttt{/usr/local/IDQ/Quantis}\\ \texttt{mkdir /usr/local/IDQ}\\
    \texttt{tar xjvf QuantisRNG-2.5.0-Linux-amd64.tar.bz2}\\ \texttt{mv
    QuantisRNG-2.5.0-Linux-amd64 /usr/local/IDQ/Quantis}\\ After this step the
    \texttt{EasyQuantis} application should be in\\
    \texttt{/usr/local/IDQ/Quantis/bin/EasyQuantis}
    \item Copy {Quantis.h} and {DllMain.h} from\\
    \texttt{\QuantisDistVersion /Libs-Apps/Quantis}\\ to\\
    \texttt{/usr/local/IQD/Quantis/include}
    \item Add \texttt{/usr/local/IDQ/Quantis/lib} (or
    \texttt{/usr/local/IDQ/Quantis/lib64}) to \\ \texttt{/etc/ld.so.config} or
    some file which is included by this file \eg\ \\
    \texttt{/etc/ld.so.conf.d/quantis.conf}
    \item Update the loader cache\\
    \texttt{ldconfig}
\end{enumerate}

\end{document}
