\pdfoutput=1
% v. 0.01 - 23/04/2012 - initial version prepared during preliminary tests 
% v. 0.02 - 07/05/2012 - some functions described, language corrections

\documentclass[a4paper,11pt]{article}

\usepackage{fullpage}
\usepackage{graphicx}
\usepackage{hyperref}
\usepackage{listings}
\usepackage{dsfont} % some math symbols
\usepackage{courier} % ttfamily
\usepackage[utf8]{inputenc}

\renewcommand{\familydefault}{\sfdefault}
\newcommand{\eg}{\textsl{e.g}}

\lstset{language=C}
\lstset{captionpos=b,basicstyle=\ttfamily,frame=lb,mathescape=true}

\title{Introduction to \texttt{libQRNG} library}
\author{Jaros{\l}aw A. Miszczak\\
Institute of Theoretical and Applied Informatics, Polish Academy of Sciences\\
Baltycka 5, 44-100 Gliwice, Poland}
\date{07/05/2012 (v. 0.02)}

\begin{document}

\maketitle

\textbf{\underline{Note}}: Functions provided by \lstinline{libQRNG} are
described in \lstinline{libQRNG.h} header file.

%%%%%%%%%%%%%%%%%%%%%%%%%%%%%%%%%%%%%%%%%%%%%%%%%%%%%%%%%%%%%%%%%%%%%%%%%%%%%%%%
\section{Establishing the connection with the QRGN server}
%%%%%%%%%%%%%%%%%%%%%%%%%%%%%%%%%%%%%%%%%%%%%%%%%%%%%%%%%%%%%%%%%%%%%%%%%%%%%%%%

In order to retrieve data from the QRNG service provided by the PicoQuant and
the Department of Physics of Humboldt University, it is necessary to register on
the service web page~\cite{qrng-www}. In order to establish a connection with
the server one can use one of the functions
\begin{itemize}
    \item \lstinline{qrng_connect(qrng_username, qrng_password)} -- establish a
      connection with the QRNG service (qrng.physik.hu-berlin.de:4499),

    \item \lstinline{qrng_connect_SSL(qrng_username, qrng_password)} --
      establish a secure connection with the QRNG service using SSL.
\end{itemize}
Both functions require a username and a password provided during the
registration. The connection is closed using \lstinline{qrng_disconnect()}
function.
\begin{itemize}
	\item \lstinline{qrng_disconnect()} -- close the connection with the QRNG
	server.
\end{itemize}

%%%%%%%%%%%%%%%%%%%%%%%%%%%%%%%%%%%%%%%%%%%%%%%%%%%%%%%%%%%%%%%%%%%%%%%%%%%%%%%%
\section{Retrieving random data}
%%%%%%%%%%%%%%%%%%%%%%%%%%%%%%%%%%%%%%%%%%%%%%%%%%%%%%%%%%%%%%%%%%%%%%%%%%%%%%%%
QRNG service allows to retrieve random data in the form of integer numbers,
double precision numbers or bytes. Integer and double precision numbers can be
retrieved in form of single values or arrays, while bytes can be retrieved only
in arrays.

%%%%%%%%%%%%%%%%%%%%%%%%%%%%%%%%%%%%%%%%%%%%%%%%%%%%%%%%%%%%%%%%%%%%%%%%%%%%%%%%
\subsection{Base functions}
%%%%%%%%%%%%%%%%%%%%%%%%%%%%%%%%%%%%%%%%%%%%%%%%%%%%%%%%%%%%%%%%%%%%%%%%%%%%%%%%
The following two functions allow to retrieve single numbers of type
\lstinline{int} or \lstinline{double}:
\begin{itemize}
    \item \lstinline{int qrng_get_int(int *val)} -- get single \lstinline{int}
    and put at \lstinline{val},
    \item \lstinline{int qrng_get_double(double *val)} -- get single
    \lstinline{double} and put at \lstinline{val}.
\end{itemize}

%%%%%%%%%%%%%%%%%%%%%%%%%%%%%%%%%%%%%%%%%%%%%%%%%%%%%%%%%%%%%%%%%%%%%%%%%%%%%%%%
\subsection{Functions for operating on arrays}
%%%%%%%%%%%%%%%%%%%%%%%%%%%%%%%%%%%%%%%%%%%%%%%%%%%%%%%%%%%%%%%%%%%%%%%%%%%%%%%%
The following functions allow to retrieve arrays of numbers of type
\lstinline{int} or \lstinline{double}.

\begin{itemize}
    \item \lstinline{int qrng_get_int_array(int *arr, int dim, int *ac)} -- get
    \lstinline{dim} integer numbers and put them into \lstinline{arr} array
    \item \lstinline{int qrng_get_double_array(double *arr, int dim, int *ac)}
    -- get \lstinline{dim} double precision numbers and put them into
    \lstinline{arr} array
\end{itemize}

The above functions report the actual number of retrieved numbers and put this
information in~\lstinline{ac}.

\begin{thebibliography}{2}
    \bibitem{qrng-www} QRNG Service, \url{http://qrng.physik.hu-berlin.de}

    \bibitem{wahl11ultrafast} M.~Wahl, M. Leifgen, M. Berlin, T. R\"ohlicke,
       H.-J. Rahn, and O. Benson, \emph{An ultrafast quantum random number
       generator with provably bounded output bias based on photon arrival time
       measurements}, Appl. Phys. Lett. 98, 171105 (2011)
       \url{http://dx.doi.org/10.1063/1.3578456}
\end{thebibliography}

\end{document}
